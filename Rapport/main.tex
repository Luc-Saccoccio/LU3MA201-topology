%! TeX program = lualatex
\documentclass[a4paper, 12pt]{article}

\usepackage{fontspec}
\usepackage{multicol}
\usepackage[french]{babel}
\usepackage[top=2.5cm, bottom=2.5cm, left=2.5cm, right=2.5cm]{geometry}
\usepackage{enumitem}
\usepackage{graphicx}
\graphicspath{ {images/} }
\usepackage{wrapfig}
\usepackage[french]{datetime}
\usepackage{amsmath}
\usepackage{amssymb}
\usepackage{amsfonts}
\usepackage{stmaryrd}
\usepackage{amsthm}
\usepackage{array}
\usepackage{hyperref}
\usepackage{minted}
\usepackage{newunicodechar}
\usepackage{emoji}

% Required for UTF-8 characters in minted env
\setmonofont{FiraCode}

\usemintedstyle{tango}
\setminted[lean]{linenos, breaklines, fontsize=\footnotesize}
\setminted[Lean]{breaklines, fontsize=\footnotesize} % For tabular
\setminted[haskell]{linenos, fontsize=\footnotesize}
\newcommand{\lean}[1]{\mintinline[breaklines]{lean}{#1}}
\newcommand{\haskell}[1]{\mintinline{haskell}{#1}}

\newunicodechar{∘}{\circ}
\newunicodechar{▸}{\blacktriangleright}
\newunicodechar{↦}{\mapsto}
\newunicodechar{ᶜ}{\textsuperscript{c}}
\newunicodechar{ᵉ}{\textsuperscript{e}}
\newunicodechar{ᵤ}{\textsubscript{u}}
\newunicodechar{ᵥ}{\textsubscript{v}}
\newunicodechar{⋂}{\ensuremath{\bigcap}}
\newunicodechar{⋃}{\ensuremath{\bigcup}}
\newunicodechar{🎉}{\emoji{party-popper}}

\hypersetup{
    colorlinks=true,
    linkcolor=blue,
    pdftitle={Démonstration assistée par ordinateur},
}

\urlstyle{same}

\begin{document}

\begin{titlepage}

    \begin{figure}
        \centering
        % \includegraphics[width=8cm]{logo.png} % Logo Sorbonne
        \vspace{8cm}
    \end{figure}

    \vspace{6cm}

    \begin{center}
        {\Huge Démonstration assistée par ordinateur\\}
        \vspace{0.5cm}
        {\huge Rapport final\\}

        \vspace{1.5cm}

        {\large Luc Saccoccio--Le Guennec\\ Nour Bouabid\\ Charles De Vergeron}
        \vfill
        {\large \today}
        \pagebreak
    \end{center}

\end{titlepage}

\tableofcontents

\newpage

\section{Introduction}

\subsection{Assistants de preuves}

Un assistant de preuve est un logiciel permettant l'écriture d'énoncés logiques et la vérification de preuves formelles. Il existe de nombreux assistants de preuves, parmi lesquels on peux citer :

\begin{itemize}
    \item Coq, développé en OCaml par l'équipe \href{http://www.pps.univ-paris-diderot.fr/pi.r2/}{PI.R2}
    \item Isabelle, développé en Standard ML et Scala
    \item Agda, développé en Haskell
\end{itemize}

On peut noter qu'ils diffèrent sur de nombreux points. En prenant Coq comme standard :

\begin{itemize}
    \item Isabelle n'est pas en capacité de raisonner avec des types dépendants
    \item Agda ne dispose pas de mécanismes d'automatisation de preuves
\end{itemize}

Qui plus est, il existe des langages de programmation dont le système de types permet la formalisation de certains énoncés, comme Haskell; ou n'ayant pas pour objectif de permettre la formalisation d'énoncés mathématiques mais plutôt la vérification de programmes informatiques, comme F*.

Le fonctionnement des assistants de preuves, autrement dit ce qui permet de s'assurer que l'assistant comprend bien les énoncés que nous encodons et qu'il a en effet raison lorsqu'il affirme que notre preuve est (ou n'est pas) acceptable; repose sur la, ou plutôt une théorie des types.

\subsection{Petit historique}

Ce qui suit est un historique extrêmement concis et réducteur de plus de 2 000 ans d'histoire d'une certaine branche de la logique.

L'histoire de la formalisation des mathématiques commence avec Aristote et l'introduction de la méthode axiomatique et de règles logiques. Après de nombreuses péripéties, on peut considérer que ces objectifs furent atteint par Frege pour la logique classique, et par Heyting pour la logique intuitionniste. Cette considération est néanmoins discutable, et en 1910 Whitehead et Russell introduisirent la notion de type pour y remédier dans les volumineux \textit{Principia Mathematica}. Malgré son efficacité à résoudre les problèmes ayant motivé sa création, c'est à dire les paradoxes de la théorie naïve des ensembles de Frege, cette théorie était assez rigide et ne formalisait pas le mécanisme de substitution. C'est avec le $\lambda$-calcul d'Alonzo Church en 1940 que ce mécanisme est formalisé, et il crée la théorie des types simples. En 1934 puis en 1958, Haskell Curry remarque qu'il y a une forme de correspondance entre les types et les schémas d'axiomes intuitionnistes, puis de même entre certains modèles de preuves et certains modèles de calcul. Enfin, en 1969, William Alvin Howard observe que la déduction naturelle peut s'interpréter comme sa version intuitionniste vu comme une version typée du $\lambda$-calcul :

\begin{center}
    Il y a un isomorphisme entre les $\lambda$-termes et les preuves, et entre leurs types et les énoncés logiques.
\end{center}

C'est la correspondance de Curry-Howard(-Lambek pour la version catégoriste)

Dans les années 70, de Bruijn étend le système de types en introduisant les types dépendants avec pour objectif affiché de formaliser et vérifier les mathématiques. Il crée alors le premier assistant de preuve, \textbf{AUTOMATH}. En 1973, un autre assistant de preuve, Mizar, est développé, mais sur des bases ensemblistes (en particulier la théorie des ensembles de Tarski-Grothendieck). Il est publié en 1989. Mais à partir des années 80, il y a une explosion de la recherche en ce domaine, avec Per Martin-Löf, Girard et Barendregt...

L'histoire est parcellée de péripéties, mais nous finirons cet historique en 2004 avec la formalisation en Coq du théorème des quatres couleurs. Bien que ce résultat fut prouvé maintes fois (en 1976 par Appel et Haken, 1996, 1997...) y compris sur des assistants de preuves, la formalisation en Coq réduit d'une part les programmes à un seul : le noyau de Coq, et d'autre part automatise les très nombreux configurations à prendre en compte.

\subsection{Ce projet}

Ce projet a pour but d'implémenter une partie de notre cours de topologie métrique (LU3MA260) dans l'assistance de preuve \href{https://lean-lang.org/}{Lean}. Cela implique donc de commencer par définir les objets étudiés (espaces métriques, boules, applications continues...) puis d'encoder et de vérifier nos théorèmes et preuves.

\newpage
\section{Lean}

Pour ce projet, nous utiliserons l'assistant de preuves Lean (version 4.5.0 lors de l'écriture de ce rapport). Il y a plusieurs raisons pour ce choix parmi lesquelles :
\begin{itemize}
    \item L'environnement de développement interactif
    \item Le support natif des caractères unicode
    \item Les capacités de méta programmation
\end{itemize}

Outre, bien entendu, son utilisation dans D$\exists\forall$DUCTION \cite{deaduction}, sa simplicité d'apprentissage, la documentation (\cite{theorem_proving}, \cite{math_in_lean}) et la communauté Lean sur Zulip pouvant répondre à nos interrogations.

Nous utilisons essentiellement Mathlib, une librairie contenant un vaste ensemble de résultats mathématiques, pour des structures usuelles comme les réels ainsi que pour conclure des preuves nécessitant la manipulation d'inégalités, cela n'était pas très intéressant.

\subsection{Histoire}

Lean est un langage open-source développé depuis 2013 basé sur le calcul des constructions (la version du lambda-calcul la plus "forte" selon la classification de Barendregt \cite{Barendregt_1991}), tout comme Coq. Bien que les premiers essais aient contenus des fondations alternatives (notamment la théorie homotopique des types), en 2017 sort Lean 3, la première version stable implémentée majoritairement en C++. En 2021 commence Lean 4, une réimplémentation de Lean majoritairement en Lean avec certaines parties en C++. Cette version contient en outre un système de macros extrêmement puissant qui permet d'implémenter et de modifier un ensembles de composants du frontend en Lean plutôt qu'en C++.

\subsection{Cycle de développement}

\subsubsection{Arborescence}

Un projet Lean commence par la création d'un dossier dédié à l'aide du programme \lean{lake}. On peut alors préciser que l'on compte utiliser Mathlib dans le fichier \lean{lakefile.lean}. Ensuite on peut commencer à programmer ou prouver des résultats. Seul le second cas nous intéresse. L'arborescence d'un projet Lean consiste en un unique dossier portant le nom du projet, lequel contient un ensemble de dossiers et de fichiers Lean. L'arborescence d'un projet est un premier outil pour structurer un ensemble de résultats, on choisit de diviser les résultats en plusieurs fichiers et dossiers suivant le plus souvent les champs ou les domaines plus spécifiques auxquels ils appartiennent. Par exemple, si on souhaitait élaborer la topologie depuis les axiomes ensemblistes, on devrait développer un ensembles de lemmes accessoires en logique (n'étant pas dans la bibliothèque standard), les bases de la théorie des ensembles et enfin les bases de la topologie générale (en oubliant la construction des ensembles usuels), une arborescence possible serait :

\newpage
\begin{minted}{text}
Project
├── Logic
│   └── Basic.lean
├── Set
│   ├── Basic.lean
│   ├── Functions.lean
│   └── Indexed.lean
└── Topology
    ├── Basic.lean
    ├── Separation
    │   └── Basic.lean
    ├── Compactness
    │   └── Basic.lean
    ...
\end{minted}

\subsubsection{Structure d'un fichier}

Un fichier Lean suit le plus souvent une certaine structure :

\begin{minted}{lean}
import ...

universe ...

variable ...

...

section ...

end ...
\end{minted}

Tout d'abord, les importations de fichiers. L'idée est la même qu'un \mintinline[fontsize=\footnotesize]{python}{import} en Python ou \mintinline[fontsize=\footnotesize]{c}{#include} en C, cette commande importe l'ensemble des résultats du fichier voulue dans le fichier dans lequel se trouve la commande. Le nom du fichier à importer correspond à son chemin dans l'arborescence. Pour reprendre l'arborescence ci-dessus, on peut par exemple importer les résultat en logique avec \lean{import Project.Logic.Basic}.

On déclare ensuite un ensemble d'\textit{univers}. Un univers est un type qui classifie un ensemble de types. Les plus communs sont \lean{Type} et \lean{Prop}, le premier classifiant un ensemble de types "ordinaires", comme \lean{Nat}, \lean{String} ou \lean{Int → Nat}, et le second classifiant des propositions, comme \lean{2 = 3} et \lean{∀ x y z : ℕ, (x * y) * z = x * (y * z)}. Or il existe de nombreux autres univers puisque par exemple, \lean{Type} n'est classifié ni par \lean{Prop}, ni par lui même. En effet, dans les versions de la théorie des types dépendants dans lesquelles il est possible d'assigner \lean{Type} comme type à \lean{Type}, on aboutit à la construction d'un terme de type \lean{∀ p : *, p} qui par la correspondance de Curry-Howard, signifie que toute proposition est démontrable, rendant la théorie inconsistante. Cette contradiction est nommée paradoxe de Girard \cite{coquand}, et c'est l'équivalent en théorie des types du paradoxe de Russell en théorie des ensembles. Il existe donc une classification des univers avec des entiers naturels : \lean{Type} est de type \lean{Type 1}, qui lui même est de type \lean{Type 2}, et ainsi de suite. Le type d'une fonction est donc l'univers le plus petit dans lequel tout ses arguments existent : \lean{Nat → Nat} a pour type \lean{Type} et \lean{Type 3 → Type 5} a pour type \lean{Type 6}, à l'exception des fonctions ayant pour type de retour \lean{Prop} qui sont toujours de type \lean{Prop}. Déclarer un univers signifie donc choisir un univers quelconque et démontrer l'ensemble des résultats au sein de cet univers. Cela permet de généraliser un grand nombre de résultats indépendamment de l'univers choisi.

On peut aussi déclarer un ensemble de variables avec \lean{variable}. C'est équivalent en mathématiques à dire "On se donne ...". Il est possible d'ajouter des instances de classe aux types choisis, par exemple pour un monoïde :

\begin{minted}{lean}
variable {M : Type u} [Monoid M]
\end{minted}

On peut alors commencer à énoncer des définitions, des résultats et les démontrer. On peut aussi définir une \lean{section} (potentiellement anonyme), c'est à dire un ensemble du fichier dans lequel on pourra définir d'autres variables et univers qui n'existeront qu'au sein de cette section, et que l'on devra référencer avec \lean{Name.resultat} avec \lean{Name} le nom de la section et \lean{resultat} l'objet voulu, ou en "ouvrant" la section préalablement avec \lean{open Name}.

\subsubsection{Énoncés et preuves}

Un énoncé est un terme de type \lean{Prop}. Par exemple :

\begin{minted}{lean}
def Even (n : ℕ) : Prop := ∃ m : ℕ, n = 2 * m
\end{minted}

On peut définir des théorèmes, lemmes et exemples avec \lean{theorem}, \lean{lemma}, \lean{example}. Chaque théorème ou lemme a un nom, un type (la proposition) et une preuve :

\begin{minted}{lean}
theorem even_even : ∀ n : ℕ, Even (n + n) := sorry
\end{minted}

Chaque preuve a un but, c'est à dire la proposition à prouver, et un contexte : l'ensemble des hypothèses de la preuve. Le contexte ne contient pas les autres résultats du fichier, qui sont néanmoins utilisables. L'équivalent mathématique du contexte correspond à l'ensemble des hypothèses faites et des variables définis ("On prend ...", "On se donne ..."). Bien que certaines preuves puissent être faites sans rentrer dans un contexte de preuves (dans les cas où le résultat est une conclusion d'un autre théorème par exemple), la majorité des preuves commencent par \lean{by} qui permet de créer un contexte et d'entrer dans un environnement de preuves. On utilise ensuite un ensemble de tactiques et de résultats précédents pour prouver le but.

Étudions quelques exemples de preuves pour comprendre comment en mener une. À gauche figurera l'évolution du code de la preuve, et à droite l'évolution du contexte (qui s'affiche dans l'éditeur et donne l'avancement de la preuve à l'endroit où se trouve le curseur).

\begin{tabular}{m{0.5\textwidth} | m{0.5\textwidth}}
    \begin{minipage}[t]{0.5\textwidth}
    \begin{minted}{Lean}
example (p q : Prop) (hp : p) (hq : q) : p ∧ q := by
    \end{minted}
    \end{minipage}
    &
    \begin{minipage}[t]{0.5\textwidth}
    \begin{minted}{Lean}
p q : Prop
hp : p
hq : q
⊢ p ∧ q
    \end{minted}
    \end{minipage}
    \\\hline

    \begin{minipage}{0.5\textwidth}
    \begin{minted}{Lean}
example (p q : Prop) (hp : p) (hq : q) : p ∧ q := by
  constructor
    \end{minted}
    \end{minipage}
    &
    \begin{minipage}[t]{0.5\textwidth}
    \begin{minted}{Lean}
▶ 2 goals
case left
p q : Prop
hp : p
hq : q
⊢ p

case right
p q : Prop
hp : p
hq : q
⊢ q
    \end{minted}
    \end{minipage}
    \\\hline
    \begin{minipage}[t]{0.5\textwidth}
    \begin{minted}{Lean}
example (p q : Prop) (hp : p) (hq : q) : p ∧ q := by
  constructor
  . exact hp
  . exact hq
    \end{minted}
    \end{minipage}
    &
    \begin{minipage}[t]{0.5\textwidth}
    \begin{minted}{Lean}
▶ goals accomplished 🎉
    \end{minted}
    \end{minipage}
\end{tabular}
\vspace{\baselineskip}

On voit donc que \lean{constructor} a introduit deux sous-buts, ayant chacun son propre contexte, et que l'on prouve séparément. On use du point pour séparer syntaxiquement les deux preuves des sous-buts, afin d'améliorer la lisibilité du code, mais ils ne sont pas nécessaires pour Lean. Chaque sous-but est clôt avec une hypothèse du contexte, à l'aide de la tactique \lean{exact}.

\vspace{\baselineskip}
\begin{tabular}{m{0.6\textwidth} | m{0.4\textwidth}}
    \begin{minipage}[t]{0.6\textwidth}
    \begin{minted}{Lean}
example : ∀ a b c : Nat, a = b → a = c → c = b := by
    \end{minted}
    \end{minipage}
    &
    \begin{minipage}[t]{0.4\textwidth}
    \begin{minted}{Lean}
⊢ ∀ (a b c : Nat), a = b → a = c → c = b
    \end{minted}
    \end{minipage}
    \\\hline
    \begin{minipage}[t]{0.6\textwidth}
    \begin{minted}{Lean}
example : ∀ a b c : Nat, a = b → a = c → c = b := by
  intro a b c h₁ h₂
    \end{minted}
    \end{minipage}
    &
    \begin{minipage}[t]{0.4\textwidth}
    \begin{minted}{Lean}
a b c : Nat
h₁ : a = b
h₂ : a = c
⊢ c = b
    \end{minted}
    \end{minipage}
    \\\hline
    \begin{minipage}[t]{0.6\textwidth}
    \begin{minted}{Lean}
example : ∀ a b c : Nat, a = b → a = c → c = b := by
  intro a b c h₁ h₂
  exact Eq.trans (Eq.symm h₂) h₁
    \end{minted}
    \end{minipage}
    &
    \begin{minipage}[t]{0.4\textwidth}
    \begin{minted}{Lean}
▶ goals accomplished 🎉
    \end{minted}
    \end{minipage}
\end{tabular}
\vspace{\baselineskip}

On voit ici l'utilisation de la tactique \lean{intro}, qui introduit des hypothèses dans le contexte, en les nommant respectivement. On finit enfin avec un exemple de la tactique \lean{rw} qui réécrit le but à l'aide d'hypothèses du contexte.

\vspace{\baselineskip}
\begin{tabular}{m{0.65\textwidth} | m{0.35\textwidth}}
    \begin{minipage}[t]{0.65\textwidth}
    \begin{minted}{Lean}
example (f : Nat → Nat) (k : Nat) (h₁ : f 0 = 0) (h₂ : k = 0) : f k = 0 := by
    \end{minted}
    \end{minipage}
    &
    \begin{minipage}[t]{0.35\textwidth}
    \begin{minted}{Lean}
f : Nat → Nat
k : Nat
h₁ : f 0 = 0
h₂ : k = 0
⊢ f k = 0
    \end{minted}
    \end{minipage}
    \\\hline
    \begin{minipage}[t]{0.65\textwidth}
    \begin{minted}{Lean}
example (f : Nat → Nat) (k : Nat) (h₁ : f 0 = 0) (h₂ : k = 0) : f k = 0 := by
  rw [h₂]
    \end{minted}
    \end{minipage}
    &
    \begin{minipage}[t]{0.35\textwidth}
    \begin{minted}{Lean}
f : Nat → Nat
k : Nat
h₁ : f 0 = 0
h₂ : k = 0
⊢ f 0 = 0
    \end{minted}
    \end{minipage}
    \\\hline
    \begin{minipage}[t]{0.65\textwidth}
    \begin{minted}{Lean}
example (f : Nat → Nat) (k : Nat) (h₁ : f 0 = 0) (h₂ : k = 0) : f k = 0 := by
  rw [h₂]
  rw [h₁]
    \end{minted}
    \end{minipage}
    &
    \begin{minipage}[t]{0.35\textwidth}
    \begin{minted}{Lean}
▶ goals accomplished 🎉
    \end{minted}
    \end{minipage}
\end{tabular}
\vspace{\baselineskip}

\subsection{Méta programmation}

La capacité pour Lean d'utiliser des caractères Unicode repose sur l'introduction de syntaxe. Mais Lean est en réalité bien plus puissant. Bien que les mécanismes internes sur lesquels reposent l'ensemble des fonctionnalités de méta programmation soient trop complexe pour ce document, on pourra se référer au livre \cite{meta_lean}. Nous allons néanmoins examiner deux exemples de méta programmation en Lean (sans pour autant en explorer les technicités).

\subsubsection{Notations}

Introduire une notation permet d'écrire en Lean des mathématiques comme on les écrirait sur papier. Ainsi, une union d'ensembles $\bigcup$, une somme indexée $\Sigma$, ou même une boule ouverte centrée en $x$ d'un rayon $r$ $B(x, r)$; sont des notations qui ne sont pas natives en Lean, mais que l'on construit à l'aide des outils de méta programmation fournis. Il y a deux étapes : introduire une syntaxe puis une sémantique. Autrement dit, il faut d'abord indiquer quel pattern doit être cherché, pour en suite exprimer ce qu'il signifie. Un exemple simple :

\begin{minted}{lean}
macro x:ident ":" t:term " ↦ " y:term : term => do `(fun $x : $t => $y)
\end{minted}

Ainsi l'expression \lean{x : ℕ ↦ 2 * x} devient grâce à cette macro \lean{fun x : ℕ => 2 * x}. \lean{x} est un identifiant, \lean{t} est un type (on utilise \lean{term} pour reconnaitre un terme au sens général) et \lean{y} est une expression (éventuellement contenant l'identifiant \lean{x}). On remarque que la syntaxe et la sémantique sont introduites en une seule commande, \lean{macro}. Un autre exemple pour introduire des notations ensemblistes :

\begin{minted}{lean}
open Std.ExtendedBinder in
syntax "{" extBinder " | " term "}" : term

macro_rules
  | `({ $x:ident | $p }) => `(setFrom (fun $x:ident => $p))
  | `({ $x:ident : $t | $p }) => `(setFrom (fun $x:ident : $t => $p))
  | `({ $x:ident $b:binderPred | $p }) => `(setFrom (fun $x:ident => satisfies_binder_pred% $x $b ∧ $p))

open Std.ExtendedBinder in
macro (priority := low) "{" t:term " | " bs:extBinders "}" : term =>
  `({x | ∃ᵉ $bs:extBinders, $t = x})
\end{minted}

On introduit ici une syntaxe générale \lean{"{" extBinder " | " term "}"}, et selon la forme du premier terme, on associe une sémantique particulière à chaque cas :
\begin{itemize}
    \item \lean{{ $x:ident | $p }}, c'est le schéma de compréhension classique $\{x\ \mid\ p(x)\}$.
    \item \lean{{ $x:ident : $t | $p}}, on traite ici le cas avec une annotation de type.
    \item \lean{{ $x:ident $b:binderPred | $p }}, c'est le schéma de compréhension $\{p(x)\ \mid\ q(x)\}$. Il est interprété comme $\{x\ \mid\ p(x)\land q(x)\}$
\end{itemize}

On ajoute à cela le cas plus rare $\{f(x)\ \mid\ x\ :\ X\}$ avec l'option \lean{priority := low} pour souligner que ce cas doit être traité après avoir tenté les sémantiques précédentes, qui est équivalent à $\{y\ \mid\ \exists x\ :\ X, y = f x\}$. On peut aussi mentionner les union et intersection ensemblistes indexées par un ensemble :

\begin{minted}{lean}
open Std.ExtendedBinder in
syntax "⋂ (" extBinder "), " term:60 : term

macro_rules
  | `(⋂ ($x:ident : $t), $p) => `(iInter (fun $x:ident : $t => $p))

open Std.ExtendedBinder in
syntax "⋃ (" extBinder "), " term:60 : term

macro_rules
  | `(⋃ ($x:ident : $t), $p) => `(iUnion (fun $x:ident : $t => $p))
\end{minted}

Le principe est le même, on introduit une syntaxe puis on explicite la sémantique. Il y a néanmoins une nuance supplémentaire. Dans les schémas de compréhension d'un ensemble, il n'y a pas d'ambigüité possible lors de la reconnaissance du pattern, puisque la syntaxe est "bornée" par les caractères \lean{\{} et \lean{\}}, or dans le cas présent il n'y a pas de délimitation à droite. C'est pourquoi on précise la priorité de l'opérateur (qui permet de définir la priorité des opérations, operator precedence pour le terme anglais) avec \lean{term:60}. Une analogie en mathématique est la priorité de $\times$ sur $+$ ($\times$ a un ordre supérieur à $+$). À noter que les commandes \lean{notation} (et ses cas plus spécifiques \lean{infix}, \lean{infixl}, \lean{infixr} et \lean{prefix}) supportent aussi l'ajout du suffixe \lean{:n} avec \lean{n} l'ordre de la priorité. Ainsi, il n'y a pas d'ambigüité possible et on peut choisir (en ayant connaissance de l'ordre des opérations usuelles, comme l'application de fonctions ou \lean{∧}) ce qui est reconnu par cette syntaxe.

Il devient alors possible d'écrire \lean{⋃ (i : ι), s i} pour $\bigcup_{i\in\iota} S_i$ (puisqu'une famille d'ensembles indexés par $\iota$ est une fonction de $\iota$ dans les parties de l'union de tout les ensembles considérés) et d'utiliser les notations \lean{{2*n | n : ℕ}} ou \lean{⋃(n : ℕ), {n}} (et elles ont bien le sens voulu).

\subsubsection{Tactiques}

Les tactiques sont l'autre force de Lean. Une tactique est un mécanisme d'automatisation d'un certain nombre d'étapes de la preuve. Parmi les tactiques usuelles, on peut citer :
\begin{itemize}
    \item \lean{intro} : Elle introduit une ou plusieurs hypothèses dans le contexte (potentiellement avec un nom), et modifie le but pour enlever les hypothèses dans la chaîne d'implications.
    \item \lean{use} : Utilisée pour prouver un but du type \lean{∃x, p x}, elle fixe \lean{x} et modifie le but en \lean{p x}.
    \item \lean{exact} : Clôt un but avec une preuve ayant cette conclusion.
    \item \lean{have} : Nomme un résultat intermédiaire, qui doit néanmoins être prouvé.
    \item \lean{assumption} : Clôt un but avec une hypothèse du contexte
    \item \lean{constructor} : Applique un constructeur ("ou" logique, "et" logique, double implication...) et crée un ensemble de buts correspondants
    \item \lean{sorry} : Clôt un but sans le prouver.
\end{itemize}

La tactique \lean{sorry} ne prouve rien en tant que tel, elle marque juste la preuve comme terminé et utilisant l'instruction \lean{sorry}. Cela évite des erreurs lors de la vérification mais prévient néanmoins l'utilisateur que la preuve n'est pas terminée.

Il y a un ensemble de tactiques qui peuvent être utilisées sur des hypothèses précises en place du but :
\begin{itemize}
    \item \lean{apply} : Applique une propriété au but (peut introduire des buts secondaires).
    \item \lean{rw} : Réécrit le but avec un ou plusieurs résultats de la forme \lean{a = b} ou \lean{a ↔ b}.
    \item \lean{simp} : Simplifie(au sens large) une expression avec un ensemble de propositions choisies augmenté d'un ensemble de propositions fournies par l'utilisateur.
\end{itemize}

Par défaut seul le but est modifié, mais on peut préciser une hypothèse, un ensemble d'hypothèses ou toutes les hypothèses, ainsi que le but; que la tactique modifiera avec le suffixe \lean{at h}, \lean{at [h1, h2]}, \lean{at *} respectivement.

\vspace{\baselineskip}
Une tactique est une macro ou une commande de type \lean{tactic}. Pour les cas les plus simples, une macro suffit. Par exemple, une tactique résolvant les cas triviaux d'égalités avec plusieurs constructeurs logiques pourrait être :

\begin{minted}{lean}
syntax "custom_trivial" : tactic

macro_rules
| `(tactic| custom_trivial) => `(tactic| constructor <;> custom_trivial)

example : 42 = 42 ∨ 1 + 1 = 2 ∧ 2 ^ 2 = 2 * 2 := by custom_trivial
\end{minted}

Où \lean{<;>} est un opérateur de tactiques tel que \lean{a <;> b} applique \lean{a}, puis applique \lean{b} à tout les buts générés par \lean{a}. Les tactiques plus complexe, demandant par exemple d'agir directement sur le contexte ou la liste de buts doivent être écrites avec \lean{elab}. Nous reproduisons ici un exemple\footnote{L'essentiel est de comprendre \textit{l'idée} plutôt que le code. Les prérequis nécessaires sont détaillés dans le livre cité. Pour le lecteur intéressé, le code est néanmoins expliqué de façon plus détaillée en annexe.} tiré de Metaprogramming in Lean 4 \cite{meta_lean} :

\begin{listing}[H]
    \begin{minted}{lean}
elab "custom_assump" : tactic =>
  Lean.Elab.Tactic.withMainContext do
    let goal ←Lean.Elab.Tactic.getMainGoal
    let goalType ←Lean.Elab.Tactic.getMainTarget
    let ctx ←Lean.MonadLCtx.getLCtx
    let option_matching_expr ←ctx.findDeclM? fun decl: Lean.LocalDecl => do
      let declExpr := decl.toExpr
      let declType ←Lean.Meta.inferType declExpr
      if ←Lean.Meta.isExprDefEq declType goalType
        then return Option.some declExpr
        else return Option.none
    match option_matching_expr with
    | some e => Lean.Elab.Tactic.closeMainGoal e
    | none =>
      Lean.Meta.throwTacticEx `custom_assump goal
        (m!"unable to find matching hypothesis of type ({goalType})")
    \end{minted}
    \label{code:custom_assump}
\end{listing}

Cette tactique reproduit la tactique \lean{assumption} qui clôt un but avec une hypothèse ayant le même type. Le principe est assez simple : la tactique récupère le but pour obtenir son type, récupère le contexte et itère à travers chaque hypothèse pour en trouver une ayant le même type que le but. Si c'est le cas, la tactique clôt la preuve avec cette tactique, sinon elle prévient l'utilisateur qu'aucune hypothèse du type voulu n'a été trouvée.

\newpage
\section{La topologie métrique en Lean}

\subsection{Contexte et mise en place}

Le cours de topologie LU3MA260\cite{topology_memo} est divisé 5 parties et 2 appendices :

\begin{itemize}
    \item Espaces métriques
    \item Complétude
    \item Compacité
    \item Connexité
    \item Espaces vectoriels normés, espaces de Banach
    \item Appendice I : ensembles et applications
    \item Appendice II : quelques schémas
\end{itemize}

Nous laisserons de côté les appendices ainsi que les deux dernières sections sur la connexité, les espaces vectoriels normés et les espaces de Banach.

\subsection{Espaces métriques}

\subsubsection{Premières définitions}

On commence par définir un espace métrique. C'est une classe de types sur lesquels on peut définir une "distance", c'est à dire la donnée de l'application et des propriétés qu'elle vérifie :

\begin{minted}{lean}
class MetricSpace (α : Type u) where
    dist : α → α → ℝ
    dist_pos : ∀ {x y : α}, 0 ≤ dist x y
    dist_sep : ∀ {x y : α}, dist x y = 0 ↔ x = y
    dist_symm : ∀ {x y : α}, dist x y = dist y x
    dist_triangle : ∀ {x y z : α}, dist x z ≤ dist x y + dist y z

variable {X : Type u} [MetricSpace X]
\end{minted}

La dernière instruction, \lean{variable}, permet de définir dans le contexte (soit le fichier, soit la \lean{section} dans laquelle on se trouve) des variables. Ici, nous nous sommes donnés un type \lean{X} dans l'univers \lean{u}, ainsi qu'une instance de la classe \lean{MetricSpace} pour le type \lean{X}. En termes mathématiques, nous nous sommes donnés un ensemble $X$ ainsi qu'une structure d'espace métrique sur $X$.

On définit aussi, car cela sera utile dès que l'on manipulera plusieurs espaces métriques distincts, deux notations plus pratiques pour la distance :

\begin{minted}{lean}
notation "d" => dist
notation "d_[" α "]" => @dist α
\end{minted}

On définit ensuite les boules ouvertes et on introduit la notation usuelle $B(x, r)$ pour la boule de rayon $r$ centrée en $x$. Le type \lean{Set X} est le type des ensembles à valeurs dans \lean{X}.

\begin{minted}{lean}
def open_ball (x : X) (r : ℝ) : Set X := {y | d x y < r}

notation "B(" x "," r ")" => open_ball x r
\end{minted}

Il est très pratique d'introduire un ensemble de lemmes relativement évidents qui faciliteront les preuves. Ils sont tous marqués avec \lean{@[simp]}, ce qui indique à Lean qu'il peut les utiliser lors de l'application des tactiques \lean{simp} :

\begin{minted}{lean}
@[simp]
lemma dist_sep_eq_zero (x : X) : d x x = 0 := dist_sep.mpr rfl

@[simp]
lemma mem_open_ball (x : X) (r : ℝ) (y : X) : y ∈ B(x, r) ↔ dist x y < r := Iff.rfl

@[simp]
lemma center_in_ball (x : X) (r : ℝ) : r > 0 → x ∈ B(x, r) := by
  intro r_pos
  simp [open_ball]
  assumption
\end{minted}

Dans le premier lemme, il suffit de considérer seulement l'une des implications de l'axiome \lean{dist_sep} en fournissant la preuve que \lean{x = x} avec \lean{rfl}. Le second repose essentiellement sur la définition d'une boule ouverte et sur la définition d'un ensemble en Lean. Enfin, le dernier lemme consiste en l'introduction de l'hypothèse de positivité de \lean{r}, puis la simplification de la condition d'appartenance à l'aide des deux lemmes précédents (ce qui justifie l'usage de \lean{@[simp]}). Le but devient \lean{0 < r}, ce qui est vrai par hypothèse, d'où l'usage de la tactique \lean{assumption}.

On finit enfin par la définition d'un ouvert. C'est un objet de type \lean{Prop}, autrement dit un énoncé mathématique, paramétré par l'ouvert $U$.

\begin{minted}{lean}
def is_open (U : Set X) : Prop := ∀ x ∈ U, ∃ r > 0, B(x, r) ⊆ U
\end{minted}

On peut alors montrer un premier résultats, c'est à dire que le boule ouverte est bien ouverte :

\begin{minted}{lean}
lemma open_ball_is_open : ∀ x : X, ∀ r > 0, is_open B(x, r) :=
  by
    intros x r _ y y_in
    set ε := r - d x y with hε
    use ε
    constructor
    . simp [open_ball] at y_in
      linarith only [hε, y_in]
    . intros z z_in
      rw [mem_open_ball] at *
      have p : d x z ≤ d x y + d y z := dist_triangle
      linarith only [p, z_in, y_in, hε]
\end{minted}

Pour ce qui est de la preuve, elle suit le raisonnement suivant :

\begin{itemize}
    \item On introduit les variables \lean{x}, \lean{r}, \lean{y} et l'hypothèse \lean{y_in} que \lean{y} est dans la boule \lean{B(x, r)}.
    \item On défini le bon \lean{ε} et l'hypothèse \lean{hε} qui contient la définition de \lean{ε}, et on l'utilise pour démontrer que la boule est ouverte.
    \item On prouve alors les deux buts :
        \begin{itemize}
            \item On montre que \lean{ε > 0} à l'aide de la définition de la boule ouverte et de \lean{linarith}
            \item On montre l'inclusion des boules à l'aide de la définition par la distance d'une boule et de l'inégalité triangulaire, et on conclue avec \lean{linarith}
        \end{itemize}
\end{itemize}

On démontre ensuite trois lemmes qui prouvent que les ouverts définis par les boules forment bien une topologie :

\begin{minted}{lean}
lemma union_open_is_open (I : Set (Set X)) : (∀ U ∈ I, is_open U) → is_open (⋃₀ I) :=
  by
    intro U_open x ⟨U, U_app_I, x_app_U⟩
    obtain ⟨r, r_pos, ball_in_U⟩ : ∃ r > 0, B(x, r) ⊆ U :=
      (U_open U) U_app_I x x_app_U
    use r, r_pos
    . exact Set.Subset.trans ball_in_U (Set.subset_sUnion_of_mem U_app_I)

lemma inter_open_is_open : ∀ U V : Set X, is_open U → is_open V → is_open (U ∩ V) :=
  by
    intro U V U_open V_open x ⟨x_in_U, x_in_V⟩
    obtain ⟨rᵤ, rᵤ_pos, ball_in_U⟩ : ∃ rᵤ > 0, B(x, rᵤ) ⊆ U := U_open x x_in_U
    obtain ⟨rᵥ, rᵥ_pos, ball_in_V⟩ : ∃ rᵥ > 0, B(x, rᵥ) ⊆ V := V_open x x_in_V
    use min rᵤ rᵥ, lt_min rᵤ_pos rᵥ_pos
    . intro y y_in_ball
      simp at y_in_ball
      apply And.intro
      . exact ball_in_U y_in_ball.left
      . exact ball_in_V y_in_ball.right

lemma space_open : is_open (Set.univ : Set X) :=
  fun x _ => ⟨1, (And.intro zero_lt_one (Set.subset_univ B(x, 1)))⟩
\end{minted}

Le premier montre que l'union d'un ensemble quelconque d'ouvert est ouvert. La preuve est immédiate, il suffit de prendre un élément dans l'union et prendre la boule dans l'ouvert dans lequel se trouve l'élément, avant de conclure par transitivité de $\subseteq$. Le second montre que l'intersection de deux ouvert est ouverte. Il faut alors prendre deux boules dans chacun des ouverts puis on prend la boule de rayon minimal. Il suffit alors de démontrer que cette boule est incluse dans chacun des ouverts, ce qui est immédiat. Enfin, on démontre que l'espace tout entier est ouvert en prenant une boule de rayon 1.

On définit enfin l'intérieur d'un ensemble comme l'union des ouverts inclus, qui est donc ouvert par le lemme montré plus haut. On note l'utilisation de \lean{simp_all} qui évite d'écrire \lean{(fun _ u_in => u_in.left)}.

\begin{minted}{lean}
def Interior (E : Set X) := ⋃₀ {U : Set X | is_open U ∧ U ⊆ E}
lemma interior_is_open {E : Set X} : is_open (Interior E) := union_open_is_open {U : Set X | is_open U ∧ U ⊆ E} (by simp_all)
\end{minted}

On démontre enfin l'équivalence entre la définition topologique et la définition métrique de l'intérieur :

\begin{minted}{lean}
@[simp]
lemma metric_interior {E : Set X} {x : X} : x ∈ Interior E ↔ ∃ r > 0, B(x, r) ⊆ E :=
  by
    constructor
    . intro x_in_E
      rcases x_in_E with ⟨U, ⟨U_open, U_sub_E⟩, x_in_U⟩
      obtain ⟨r, r_pos, ball_in_U⟩ : ∃ r > 0, B(x, r) ⊆ U := U_open x x_in_U
      use r, r_pos
      trans U <;> assumption
    . rintro ⟨r, r_pos, ball_in_E⟩
      have ball_open : is_open B(x, r) := open_ball_is_open x r r_pos
      use B(x, r)
      simp_all
\end{minted}

On procède par double implication, en se donnant un élément de l'espace
\begin{itemize}
    \item Si l'élément est dans l'intérieur, il est dans un ouvert inclus dans E, donc il existe une boule ouverte centrée en ce point, incluse dans l'ouvert. Or cet ouvert est inclus dans E donc par transitivité, la boule est dans E.
    \item S'il existe une boule centrée en l'élément, puisqu'une boule est ouverte, la boule est dans l'intérieur, est l'élément étant dans la boule, l'élément est dans l'intérieur.
\end{itemize}

\subsubsection{Continuité}

On commence par définir la continuité en un point, puis la continuité globale :

\begin{minted}{lean}
def continuous_on (f : X → Y) (x₀ : X) : Prop :=
  ∀ ε > 0, ∃ δ > 0, ∀ x, d_[X] x₀ x < δ → d_[Y] (f x₀) (f x) < ε

def Continuous (f : X → Y) : Prop := ∀ x : X, continuous_on f x
\end{minted}

On démontre ensuite l'équivalence entre la définition métrique et la définition topologique de la continuité :

\begin{minted}{lean}
theorem topologic_continuity (f : X → Y) : Continuous f ↔ (∀ U, is_open U → is_open (f ⁻¹' U)) :=
  by
    constructor
    . intro h U U_open x₀ x₀_in_reci_f
      obtain ⟨ε, ε_pos, ball_in_U⟩ : ∃ ε > 0, B(f x₀, ε) ⊆ U := U_open (f x₀) x₀_in_reci_f
      rcases (h x₀) ε ε_pos with ⟨δ, δ_pos, H⟩
      use δ, δ_pos
      intro x hx
      suffices hh : f x ∈ B(f x₀, ε) from ball_in_U hh
      exact H x hx
    . intro H₁ x₀ ε ε_pos
      let U : Set Y := B(f x₀, ε)
      have U_open := open_ball_is_open (f x₀) ε ε_pos
      have recU_open := H₁ U U_open
      have x_in_recU: x₀ ∈ f⁻¹' U := by simpa
      obtain ⟨δ, δ_pos, H₂⟩ : ∃ δ > 0, B(x₀, δ) ⊆ f⁻¹' U := recU_open x₀ x_in_recU
      use δ, δ_pos
      intro x hx
      exact H₂ hx
\end{minted}

On procède une nouvelle fois par double implication :
\begin{itemize}
    \item On veut montrer que l'image réciproque d'un ouvert par une fonction continue est ouverte.
        \begin{itemize}
            \item On prend donc un ouvert, un point dans l'image réciproque de cet ouvert ainsi qu'une boule ouverte dans l'ouvert.
            \item Par l'hypothèse de continuité, on obtient alors \lean{δ} tel que \lean{∀ (x : X), d x₀ x < δ → d (f x₀) (f x) < ε}. On utilise ce \lean{δ} comme rayon de la boule dans l'image réciproque
            \item Il suffit alors de prouver que l'image de tout point de la boule dans l'image réciproque est dans la boule de l'ouvert de départ, ce qui est vrai par continuité de la fonction
        \end{itemize}
    \item Réciproquement, on veut montrer la continuité de la fonction lorsqu'elle conserve les ouverts par image réciproque.
        \begin{itemize}
            \item On se donne donc un point \lean{x₀} et un \lean{ε} (strictement positif). On prend alors comme ouvert la boule centrée en \lean{f x₀}, dont l'image réciproque est ouverte par hypothèse.
            \item Il existe alors une boule de rayon \lean{δ} dans l'image réciproque, on peut donc prendre ce \lean{δ} et on a bien la continuité de la fonction.
        \end{itemize}
\end{itemize}

On montre ensuite, à l'aide du critère topologique, que la composée de deux applications continues est continue :

\begin{minted}{lean}
theorem comp_continuous (f : X → Y) (g : Y → Z) : Continuous f → Continuous g → Continuous (g ∘ f) :=
  by
    intro f_cont g_cont
    rw [topologic_continuity] at *
    intro V V_open
    let U := g⁻¹' V
    have U_open : is_open U := g_cont V V_open
    exact f_cont U U_open
\end{minted}

On se donne deux fonctions et les hypothèses de continuité, on réécrit le but et les hypothèses à l'aide de l'équivalence entre les définitions métrique et topologique, on se donne un ouvert et il suffit d'en prendre l'image réciproque par \lean{g} (qui est ouvert par continuité de \lean{g}) puis l'image réciproque par \lean{f} de l'image réciproque par \lean{g}, qui est ouverte par continuité de \lean{f}.

On finit cette section en étudiant le cas des fonctions lipschitziennes :

\begin{minted}{lean}
def Lipschitz (f : X → Y) : Prop :=
  ∃ k > 0, ∀ x x' : X, d_[Y] (f x) (f x') ≤ k * d_[X] x x'

theorem lipschitz_implies_continuous (f : X → Y) : Lipschitz f → Continuous f :=
  by
    intro ⟨k, k_pos, h⟩
    intro x₀ ε ε_pos
    use ε / k, div_pos ε_pos k_pos
    intro x hdx
    have h₁ : d_[Y] (f x₀) (f x) < k * (ε / k) := lt_of_le_of_lt (h x₀ x) (mul_lt_mul_of_pos_left hdx k_pos)
    rw [mul_div_cancel' ε] at h₁
    . exact h₁
    . linarith
\end{minted}

On se donne l'hypothèse de lipschitziannité de \lean{f}, ainsi qu'un point quelconque \lean{x₀} et un \lean{ε}. On veut montrer la continuité en \lean{x₀}. On choisit donc \lean{δ := ε / k}, qui est bien strictement positif car quotient de réels strictement positifs. On conclut en montrant que l'on a bien \lean{d_[Y] (f x₀) (f x) < ε}.

\subsubsection{Fermés}

On finit en définissant les fermés et l'adhérence. Commençons par les fermés :

\begin{minted}{lean}
def is_closed (F : Set X) := is_open Fᶜ

lemma union_closed_is_closed : ∀ F G : Set X, is_closed F → is_closed G → is_closed (F ∪ G) :=
  by
    intro F G F_closed G_closed
    rw [is_closed, Set.compl_union F G]
    exact inter_open_is_open Fᶜ Gᶜ F_closed G_closed

lemma inter_closed_is_closed (I : Set (Set X)) : (∀ F ∈ I, is_closed F) → is_closed (⋂₀ I) :=
  by
    intro h
    rw [is_closed, Set.compl_sInter I]
    have h₁ : ∀ U ∈ compl '' I, is_open U :=
      by
        intro U hU
        simp [Set.image] at hU
        obtain ⟨F, hF⟩ := hU
        rw [←hF.right]
        exact h F hF.left
    exact union_open_is_open (compl '' I) h₁
\end{minted}

Un fermé est donc le complémentaire d'un ouvert. Les deux lemmes reposent sur leurs analogues pour les ouverts. Pour le premier, il suffit de passer au complémentaire puis d'appliquer le lemme sur l'intersection de deux ouverts. Dans le second cas, on passe au complémentaire dans l'intersection quelconque et on montre que tout les ensembles sont ouverts, puis on conclue avec le lemme sur l'union quelconque d'ouverts.

Enfin, on définit l'adhérence et quelques lemmes :

\begin{minted}{lean}
def Closure (E : Set X) : Set X := ⋂₀ {F : Set X | is_closed F ∧ E ⊆ F}

@[simp]
lemma closure_is_closed {E : Set X} : is_closed (Closure E) := inter_closed_is_closed {F : Set X | is_closed F ∧ E ⊆ F} fun _ => And.left

lemma sub_closure {E : Set X} : E ⊆ Closure E :=
    Set.subset_sInter fun _ => And.right

lemma closure_sub_closed {E F : Set X} : is_closed F → E ⊆ F → Closure E ⊆ F :=
  fun hClosed hSub => Set.sInter_subset_of_mem ⟨hClosed, hSub⟩

lemma closure_monotone {A B : Set X} (h : A ⊆ B) : Closure A ⊆ Closure B :=
    closure_sub_closed closure_is_closed (Set.Subset.trans h sub_closure)

lemma closure_closed_is_closed {E : Set X} (h : is_closed E) : Closure E = E :=
  Set.Subset.antisymm (closure_sub_closed h (Set.Subset.refl E)) sub_closure

lemma closure_closure_eq_closure {E : Set X} : Closure (Closure E) = Closure E := closure_closed_is_closed closure_is_closed
\end{minted}

\subsection{Complétude}

La complétude nécessite un certain nombre de définitions , telles que les suites de Cauchy, la caractérisation séquentielle de la fermeture par exemple.

\subsubsection{Définitions}

La première chose a définir est une suite de Cauchy :

\begin{minted}{lean}
def CauchySeq (u : ℕ → X) := ∀ ε > 0, ∃ N : ℕ , ∀ m ≥ N,∀ n ≥ N, d (u m) (u n) < ε
\end{minted}

Ainsi qu'un espace complet :

\begin{minted}{lean}
def Complet (Y : Type v) [MetricSpace Y] := ∀ u : ℕ → Y, CauchySeq u → Converg u
\end{minted}

On rappelle que \lean{d} est la distance définie plus haut.


\subsubsection{Quelques Lemmes}

\textbf{Une suite convergente est de Cauchy}\\

Le lemme suivant est assez simple dans le raisonnement. Il m'a aidé dans la prise en main et la compréhension de Lean.

\begin{minted}{lean}
theorem Converg_Cauchy (u : ℕ → X) : Converg u → CauchySeq u :=
  by
    intro ⟨l, hl⟩
    intro ε ε_pos
    have Hε : 0 < ε / 2 := by linarith
    obtain ⟨N, hN⟩ := hl ε/2
    use N
    intro m hm
    intro n hn
    have hd : d (u n) (u m) ≤ d (u n) l + d l (u m) := dist_triangle
    have h₁ : d (u n) l < ε/2 := @dist_symm X _ _ _ ▸ hN n hn
    have h₂ : d (u m) l < ε/2 := @dist_symm X _ _ _ ▸ hN m hm
    have hε := add_lt_add h₁ h₂
    have h : d (u n) (u m) < ε/2 + ε/2 := lt_trans hd hε
    have E : ε/2 + ε/2 = ε := by linarith
    have H : d (u n) (u m) < ε
\end{minted}


Il s'agit de la preuve qu'une suite de Cauchy est convergente.
L'instruction \lean{intro ⟨l, hl⟩} introduit l'hypothèse de convergence en utilisant l existence de la limite \lean{l} ainsi que en la notant hl. Nous disposons maintenant de l'énoncé général : $$\forall \varepsilon \geq 0 \exists N \in \mathbb{N},\forall n \geq N, d (u n) (l) < \varepsilon$$. Nous introduisons donc la quantité $\varepsilon$ que nous divisons par 2 dans l hypothèse He pour la suite de la preuve.
Grâce a l hypothèse, nous avons un $N$ entier appliqué a cet $\varepsilon$/2 par \lean{obtain ⟨N, hN⟩ := hl ε/2 h} qui est utilisé avec \lean{use N}.

On invoque un n et m > N pour continuer de parcourir l'hypothèse pour arriver au cœur de la preuve de ce lemme. Nous avons besoin de deux énoncés

\begin{itemize}
    \item \lean{hd} qui consiste a montrer que \lean{d (u n) (u m) ≤ d (u n) l + d l (u m)} par l'inégalité triangulaire
    \item \lean{h1} et \lean{h2} utilisent l'hypothèse de convergence appliquées respectivement au \lean{n} et \lean{m} et $\varepsilon/2$
\end{itemize}

Les énoncés he et h condensent les inégalités précédentes avec successivement
\begin{itemize}
    \item $a \leq b \land c\leq d \implies a+b \leq c+d$ (\lean{add_lt_add})
    \item $a \leq b \land b \leq c \implies a\leq c$ (\lean{lt_trans}) appliquée à \lean{hd} et \lean{he}
\end{itemize}

Il ne nous reste plus qu'à préciser a Lean que \lean{ε/2 + ε/2 = ε} grâce a la tactique \lean{linarith} pour finir la preuve.

\vspace{\baselineskip}
\textbf{Une suite de Cauchy est bornée}\\

Le second lemme est le caractère bornée des suites de Cauchy :

\begin{minted}{lean}
lemma Cauchy_borne (u : ℕ → X) : (h : CauchySeq u) → Bornee u := by
    unfold CauchySeq
    unfold Bornee
    intro h
    specialize h (1: ℝ)
    have h1 : (1:ℝ) >0 := by
      simp
    specialize h h1
    obtain ⟨N, hN⟩ := h
    have fintype.D := { d (u N) (u n) | n : Fin (N-1)}
    -- have M := D.max
    intro n
\end{minted}


La tactique \lean{unfold} déploie les définitions de \lean{CauchySeq} et \lean{Bornee} dans les hypothèses et le but. Nous introduisons l'hypothèse \lean{h} qui est le fait que la suite \lean{u} est de Cauchy. Le début de \lean{h} étant une généralité, nous spécialisons le \lean{ε} en 1 pour faciliter la preuve. (En effet, sans spécifier le epsilon choisit, il faudrait utiliser l'axiome du choix). L'énonce \lean{h1} montre que 1 pris dans les réels est bien strictement positif, critère que doit valider tout choix d'epsilon. Comme précédemment, la tactique \lean{obtain ⟨N, hN⟩} donne le \lean{N} associé a 1 (vu comme un epsilon quelconque) et note \lean{hN} la suite de l'énoncé : \lean{hN : ∀ m ≥ N, ∀ n ≥ N, d (u m) (u n) < 1}. La suite de la preuve consiste en la recherche d un majorant M réel à l'ensemble des distances possibles entre les termes de la suite et un point \lean{x} de \lean{X}. Ici, nous choisissons le \lean{N}ième terme de la suite \lean{u} comme point de référence.

De deux choses l'une :
\begin{itemize}
    \item si $M \geq 1$, M majore alors tous les termes de la suite d (u N) (u n)
    \item si $M \leq  1$ , M <- 1 et 1 est un majorant de la suite d (u N) (u n)
\end{itemize}

On définit $D := \{d\ (u\ N)\ (u\ n)\ \mid\ n \leq N-1\}$, l'ensemble des distances entre \lean{u N} et tous les autres termes \lean{u} d'indices inférieurs a \lean{N}. On note \lean{M} le max de cet ensemble.

Une autre méthode, plus algorithmique, serait de faire une induction sur n < N : poser $M := d\ (u\ N)\ (u\ 0)$ puis comparer $M$ aux autres distances. Dès que $M$ est dépassé par une distance, ce nouveau majorant remplace l'ancienne valeur de $M$.

Supposons le M obtenu, il reste a majoré les termes d indices supérieures a N. Ils le sont dés le départ par l hypothèse que la suite est de Cauchy. Le majorant de la suite sera donc le maximum entre M et 1, comme annoncé.


\textbf{Une suite de Cauchy ayant une valeur d adhérence converge}\\

Enfin, voici un dernier lemme sur la convergence des suites de Cauchy, principalement utilisée dans le chapitre sur la compacité.

\begin{minted}{lean}
lemma Cauchy_val_adherence_conv (u : ℕ → X) (l : X) (f : ℕ → ℕ)
    (h1 : strictement_croissante f) (h2 : lim (u ∘ f) l) (h3 : CauchySeq u): lim u l := by
  intro ε hε
  specialize h3 (ε/2) (half_pos hε)
  specialize h2 (ε/2) (half_pos hε)
  obtain ⟨N1, hN1⟩ := h2
  obtain ⟨N2, hN2⟩ := h3
  use Nat.max N1 N2
  intro n hn
  have h : d l (u n) ≤ d l (u (f n)) + d (u (f n)) (u n) := dist_triangle
  have ineq1: Nat.max N1 N2 ≥ N1 := Nat.le_max_left N1 N2
  have ineq2: Nat.max N1 N2 ≥ N2 := Nat.le_max_right N1 N2

  specialize hN1 n (le_trans ineq1 hn)
  specialize hN2 n (le_trans ineq2 hn)
  specialize hN2 (f n) (le_trans (le_trans ineq2 hn) (stricte_croissance_geq f h1 n))
  rw [Function.comp_apply] at hN1
  rw [dist_symm] at hN2

  linarith
\end{minted}

La fonction \lean{f} supposée strictement croissante (hypothèse \lean{h1}) est la suite extractrice. L'hypothèse \lean{h2} correspond a l'hypothèse d'existence d'une valeur d'adhérence. Comme toute preuve débutant par une généralité, invoquons un epsilon strictement positif. On injecte ensuite la moitié de cette quantité dans les hypothèse \lean{h2} (la convergence de la suite extraite) et \lean{h3} (la suite \lean{u} est de Cauchy).
On obtient ainsi \lean{N1} et \lean{N2} respectivement des hypothèses \lean{h2} et \lean{h3} pour ce \lean{ε/2}. Lean a ensuite a sa disposition le maximum de \lean{N1} et \lean{N2} grâce à \lean{use Nat.max N1 N2}.

Après toutes ces invocations et spécialisations, le but de la preuve est le suivant $$\forall n \geq max (N1,N2), d\ l\ (u\ n) \leq \varepsilon$$

On introduit donc un \lean{n} plus grand que $max(N1,N2)$. Pour ce \lean{n}, on énonce \lean{h}, prouvé par l'inégalité triangulaire.
De plus, on introduit deux inégalités : $max(N1,N2) \geq N1$ $max(N1,N2) \geq N2$ (respectivement \lean{ineq1} et \lean{ineq2}).

Voici les hypothèses à retenir :
\begin{itemize}
    \item \lean{hN1 : ∀ n ≥ N1, d l ((u ∘ f) n) < ε / 2}
    \item \lean{hN2 : ∀ m ≥ N2, ∀ n ≥ N2, d (u m) (u n) < ε / 2}
\end{itemize}

Et le but de preuve est : \lean{d l (u n) < ε}.

Tout d'abord, nous prouvons que le \lean{n} invoqué au début de la preuve est effectivement supérieur à \lean{N1} et \lean{N2} pour pouvoir utiliser les hypothèses, grâce au théorème \lean{le_trans} respectivement appliqué aux couples d'hypothèses \lean{ineq1 hn} et \lean{ineq2 hn}. On rappelle que \lean{hN2} dépend de deux généralités en \lean{m > N2} et \lean{n > N2}. On souhaite spécialiser \lean{m} en \lean{f(n)} par \lean{specialize hN2 (f n)}; la suite de la ligne montre que \lean{m > N}.
On réécrit la composition de l'hypothèse \lean{hN1} pour faire correspondre les expressions : \lean{(u ∘ f) n)} et \lean{u(f n)}. Enfin, la tactique \lean{linarith} vient synthétiser les hypothèses \lean{hN1} et \lean{hN2} pour conclure au but.


\subsubsection{Théoreme du point fixe de Banach Picard}

Il reste a montrer le théorème central de ce chapitre : le théorème du point fixe contractant de Banach picard.

\begin{minted}{lean}
theorem Point_Fixe_Contractant (f : X -> X) : (Complet X ∧ contractant f ∧ ∀ u :ℕ -> X, ∀ n , f (u n) = u (Nat.succ n)) -> (∃!x:X,f x =x ∧ lim u x) :=
  by
    unfold contractant
    have C := d (u 0) (u 1)
    intro H
    obtain ⟨k, hk⟩ := H.2

    have h : ∀ n : ℕ , d (u (Nat.succ n)) (u n) < k^n*C := sorry
    -- cf exercice 50 du poly de topo

    have l : ∀ ε>0 ,∃ N , ∀ n > N ,d (u (Nat.succ n)) (u n) < ε := sorry
    -- preuve que k^n*C tend vers zero puis thm des gendarmes

    have hp : ∀ n p : ℕ , d (u n) (u (n + p)) < ((k^n)/(1-k))*C := sorry
    -- IT generalisée et recurrence sur p

    have lp : ∀ ε>0 ,∃ N ,∀ p ≥ N,∀ n ≥ N, d (u p) (u n) < ε := sorry
    -- encore une preuve de convergence vers 0

    have Cauchy : CauchySeq u := by exact lp
    have Conv : Converge u := by exact H.1 u Cauchy
\end{minted}



\subsection{Compacité}

La notion de compacité a émergé au 19 ème siècle dans le contexte du développement de l'analyse mathématique afin de mieux formaliser les notions de convergences et de continuité. Nous choisissons de la définir d'abord de manière séquentielle plus simple et adaptée aux espaces métriques. Puis nous montrerons les grandes lignes de son lien avec la continuité, la fermeture et la complétude. Enfin nous nous orienterons vers une définition ensembliste qui s'approche de celle qui étend la compacité aux espaces topologiques quelconques. Cependant nous n'aborderons pas la topologie générale qui n'est pas au programme de L3.


\subsubsection{Définitions}

\begin{minted}{lean}
variable {X Y : Type u} [MetricSpace X] [MetricSpace Y]
\end{minted}

On introduit deux espaces métriques $(X,d_{X})$ et $(Y,d_{Y})$ grâce à la commande \lean{variable}. X et Y sont munis des distances respectives \lean{d_{X}} et \lean{d_{Y}} défini par \lean{MetricSpace X} et \lean{MetricSpace Y}, ainsi le type X désignera $(X,d_{X})$ et le type Y désignera $(Y,d_{Y})$.\\

D'après le cours un espace métrique \lean{X} est compact si et seulement si de toute suite d'éléments de \lean{X} on peut extraire une sous-suite convergente. Nous définirons un compact par cette propriété. Mais avant, voici quelque lemmes et définitions préliminaires.

\begin{minted}{lean}
def strictement_croissante (f : ℕ → ℕ) : Prop := ∀ n m : ℕ, n > m -> f n > f m
def croissante (f : ℕ → ℕ) : Prop := ∀ n m : ℕ, n >= m → f n >= f m
\end{minted}

\lean{croissante} et \lean{stricte_croissante} sont les définitions respectives de la croissance et stricte croissance d'une fonction de $\mathbb{N}$ dans $\mathbb{N}$. Ensuite le lemme \lean{stricte_croissante_to_croissante} prouve qu'une fonction strictement croissante est croissante:

\begin{minted}{lean}
lemma stricte_croissante_to_croissante (f : ℕ → ℕ) : strictement_croissante f → croissante f :=
  by
    intro hf n m hnm
    apply Or.elim (Nat.eq_or_lt_of_le hnm)
    . exact Nat.le_of_eq ∘ congrArg f
    . exact Nat.le_of_lt ∘ hf n m
\end{minted}


\begin{itemize}[align=left,itemsep=20pt]
    \item[\textbf{intro:}] Cette preuve commence par l'introduction de \lean{hf} l'hypothèse de la strict croissance de \lean{f} , \lean{n} et \lean{m} deux entiers naturels et
    \lean{hnm} l'hypothèse \lean{n ≥ m}. Le but est de type \lean{f n ≥ f m}.
    \item[\textbf{apply:}] Le lemme de mathlib \lean{Nat.eq_or_lt_of_le} appliqué à \lean{hnm} montre que \lean{n ≥ m} implique "\lean{n} = \lean{m} ou \lean{n ≥ m}". Ainsi par composition avec la tactique \lean{apply Or.elim}, on obtient une disjonction de cas.
    \item[\textbf{exact:}] Nous obtenons alors deux buts. Le premier, \lean{m = n -> f n ≥ f m} se montre par la fonction \lean{congrArg}, elle prend en argument \lean{f} et renvoie que l'égalité des antécédents implique l'égalité des images. Puis le lemme de mathlib \lean{Nat.le_of_eq} élargi l'égalité des images à l'inégalité souhaitée. Le second but \lean{ n > m -> f n >= f m} se montre par \lean{hf} avant d'élargir l'inégalité strict par \lean{Nat.le_of_lt}. Ainsi dans les deux cas, la tactique \lean{exact} prouve que la propriété obtenu est exactement celle attendue.
\end{itemize}


\vspace{\baselineskip}

D'autres lemmes que nous avons codé mais que nous ne commenterons pas seront utilisés dans cette partie, ils sont accessibles dans notre répertoire git \cite{git_repo}.

Nous pouvons à présent prouver que si une suite converge alors toute suite extraite converge vers la même limite:

\begin{minted}{lean}
lemma limite_suite_extraite (K : Set X) (x : ℕ → K) (l : X) (f : ℕ → ℕ) : lim x l ∧ strictement_croissante f -> lim (x ∘ f) l :=
  by
    rintro ⟨hx, hf⟩ ε hε
    obtain ⟨N, hN ⟩ := hx ε hε
    use N
    intro n hn
    have t : f n >= f N := stricte_croissante_to_croissante f hf n N hn
    apply hN
    have sup_N: f N >= N := stricte_croissance_geq f hf N
    linarith [t, stricte_croissance_geq f]
\end{minted}

\begin{itemize}[align=left,itemsep=20pt]
    \item [\textbf{rintro:}] Dans ce contexte la tactique \lean{rintro} permet d'introduire séparément \lean{hx} (de type lim x l) et \lean{hf} (de type \lean{strictement_croissante f}), le but étant une implication, les hypothèses de départ sont introduites en premier. \lean{rintro} introduit ensuite \lean{ε} et \lean{hε} (de type \lean{ε>0} d'après l'expression du but qui commence par le quantificateur $\forall$ (voir définition de \lean{lim}).
    \item [\textbf{obtain:}] Ainsi on obtient \lean{N} un entier naturel et \lean{hN} de type \lean{∀ n ≥ N, d l ↑(x n) < ε} par \lean{hx} appliqué à \lean{ε} et \lean{hε}.
    \item [\textbf{use:}] Le but étant alors \lean{∃ N, ∀ n ≥ N, d l ↑((x ∘ f) n) < ε}, la tactique \lean{ use N} fait que le but soit de montrer que \lean{N} convient.
    \item [\textbf{intro:}] On introduit alors \lean{n} et \lean{hn} puisque le but commence encore une fois par le quantificateur $\forall$.
    \item [\textbf{apply:}] Il reste à montrer \lean{d l ↑((x ∘ f) n) < ε}. Nous appliquons \lean{ hN} au but, ainsi il reste à montrer que \lean{f n ≥ N}.
    \item [\textbf{have:}] \lean{stricte_croissante_to_croissante} démontrée précédemment appliquée à \lean{f} et \lean{hf} nous donne la croissance de \lean{f}. Puis la définition de la croissance de \lean{f} appliquée à \lean{n},\lean{N} et \lean{hn} nous donne la propriété que nous avons nommé \lean{t} par la tactique have.
    \item [\textbf{have:}] De même, \lean{ stricte_croissance_geq} appliqué à \lean{f}, \lean{hf} et \lean{N} nous donne la propriété que nous avons nommée \lean{sup_N}.
    \item [\textbf{linarith:}] Enfin, par combinaison linéaire de ces deux dernières hypothèses,la tactique \lean{linarith} résout automatiquement le but.
\end{itemize}

\vspace{\baselineskip}
\vspace{\baselineskip}

\begin{minted}{lean}
def lim' (x : ℕ → α) (l : α) [MetricSpace α]:= ∀ ε > 0, ∃ N : ℕ, ∀ n ≥ N, d l (x n) < ε
\end{minted}

Cette définition de la limite prend en argument une suite \lean{x} de $\mathbb{N}$ dans $\alpha$, un type quelconque de l'univers \lean{u} à partir duquel on construit un espace métrique grâce à la syntaxe \lean{[MetricSpace α]}. \lean{lim'} prend en second argument un élément de $\alpha$, puis elle défini la convergence de la suite \lean{x} vers cet élément.\\

Nous pouvons à présent définir un espace métrique compact:

\begin{minted}{lean}
def Compact (K : Type u) [MetricSpace K] : Prop := ∀ x : ℕ → K , ∃ f : ℕ → ℕ, ∃ l, (strictement_croissante f ∧ lim' (x ∘ f) l)
\end{minted}

Cette définition prend encore une fois un type quelconque de \lean{u} en argument et elle en fait un espace métrique. Pour toute suite \lean{x}, l'existence d'une suite extraite est par définition équivalente à l'existence d'une fonction strictement croissante de $\mathbb{N}$ dans $\mathbb{N}$ que l'on compose par \lean{x}. Ainsi, la compacité se défini par l'existence d'une telle fonction ainsi que celle d'un élément \lean{l} vers lequel la fonction composée converge.

\begin{minted}{lean}
def is_compact (K : Set X) : Prop := ∀ x : ℕ → K, ∃ f : ℕ → ℕ, ∃ l ∈ K, strictement_croissante f ∧ lim (x ∘ f) l
\end{minted}

Une partie d'un espace métrique est compact si le sous-espace métrique correspondant est compact. Étant donné que la distance se conserve pour tout sous-espace métrique, il n'est pas nécessaire de définir explicitement un nouvel espace métrique. Une partie \lean{K} de \lean{X} est un type de \lean{Set X} et tout élément de \lean{K} est effectivement un type de \lean{X}. Ainsi \lean{K} conserve les propriétés de \lean{X}. Il suffit donc d'ajuster la définition précédente en remplaçant \lean{∀ x : ℕ → X} par \lean{∀ x : ℕ → K}, ainsi que \lean{lim'} par \lean{lim}. En effet, \lean{lim} prend en argument une suite à valeur dans un type de \lean{Set X} et \lean{lim'} ne conviens qu'à une suite à valeurs dans \lean{X}.

\subsubsection{Résultats liés à la fermeture, la continuité et la complétude }

Nous commençons par quelques résultats sur la fermeture. Tout d'abord, montrons qu'une partie compact de \lean{X} est fermé dans \lean{X}

\begin{minted}{lean}
lemma compact_is_closed : ∀ K : Set X, is_compact K → is_closed K :=
  by
    intro K h
    contrapose! h
    have diff : Closure K ≠ K := by
      intro absurde
      have j : is_closed (Closure K) := by exact closure_is_closed
      rw [absurde] at j
      apply h at j
      exact j

    have c : K ⊂ Closure K :=
      Set.ssubset_iff_subset_ne.mpr ⟨sub_closure, diff.symm⟩
    have l_in_closure_not_in_K : ∃ l : X, l ∈ Closure K ∧ l ∉ K :=
      Set.exists_of_ssubset c

    rcases l_in_closure_not_in_K with ⟨l, l_in_closure, l_not_in_K⟩
    obtain ⟨x, hx⟩ := (sequential_closure K l).mp l_in_closure

    intro compacite
    choose f l' hl' hf lim_l' using compacite x
    have lim_l : lim (x ∘ f) l := limite_suite_extraite K x l f ⟨hx, hf⟩
    have egalite: l = l':= by apply unicite_limite K (x ∘ f) l l' lim_l lim_l'
    rw [egalite] at l_not_in_K
    apply l_not_in_K at hl'
    exact hl'
\end{minted}

Afin d'expliquer cette preuve de manière claire et détaillée, nous la diviserons en plusieurs parties.

\begin{itemize}[itemsep=30pt]
    \item \textbf{Raisonnement par contraposée}\\
        Comme le but est une implication de la forme $ \forall x, P(x) \implies Q(x)$, la tactique \lean{intro} introduit d'abord $x$ puis $P(x)$. Ici on nomme \lean{K} une partie compact, et \lean{h} l'hypothèse de sa compacité. Le but devient alors \lean{is_closed K}.\\
        La tactique \lean{contrapose!} permet un raisonnement par contraposée. le but devient la négation de \lean{h} et h devient la négation du but précédent.
    \item \textbf{Construction d'une suite de \lean{K} convergeant dans \lean{X} exlu de \lean{K} }\\
        Nous nommons \lean{diff} la propriété que l'adhérence de \lean{K} est différente de \lean{K}. La tactique \lean{have} nous permet à la fois d'emboiter une "sous" preuve et de nommer la propriété prouvée. Par souci de clarté nous détaillerons la preuve de \lean{diff} dans un second temps. Nous ferons de même pour \lean{c : K ⊂ Closure K } puis \lean{l_in_closure_not_in_K}, preuve de l'existence de \lean{l} appartenant à l'adhérence de \lean{K} exclue de \lean{K}.\\
        \lean{recases} nous permet d'introduire séparément un tel \lean{l} puis les propriétés vérifiées par \lean{l}.\\
        Ensuite par la tactique \lean{obtain}, nous appliquons le lemme \lean{sequential_closure} pour \lean{K} et \lean{l} dans le sens direct (\lean{.mp}).\\
        Ainsi le critère de fermeture séquentielle nous donne l'existence d'une suite de \lean{K} convergente vers \lean{l}. \lean{obtain} nous permet d'introduire une telle suite ainsi que la propriété qu'elle vérifie puis de les nommer respectivement \lean{x} et \lean{hx}.
    \item \textbf{ Preuve que \lean{x} n'admet pas de valeur d'adhérence}\\
        Enfin, Le goal étant la négation d'une proposition, \lean{intro} introduit cette proposition et change le but en \lean{False}. Ainsi par l'absurde nous montrerons que \lean{x}, n'admet pas de valeur d'adhérence dans \lean{K}, donc que \lean{K} n'est pas compact. \\
        \lean{K} est supposé compact donc \lean{choose } applique la définition de la compacité à x, et prend alors une suite extraite de \lean{x} qui converge dans \lean{K }vers \lean{l'}, on nomme \lean{hl'} l'hypothèse que \lean{l'} est dans \lean{K}.\\
        Nous prouvons ensuite que cette suite extraite tend vers \lean{l} par le lemme \lean{limite_suite_extraite} démontrée plus haut. Puis nous prouvons que \lean{l=l'} par unicité de la limite.\\
        La tactique \lean{rw} prend en argument une égalité et en précisant \lean{at l_not_in_K } elle remplace dans \lean{l_not_in_K} le membre de gauche par le membre de droite. Ainsi \lean{l_not_in_K} devient \lean{l'} n'appartient pas à \lean{K}. Ceci est en contradiction avec \lean{hl'}.\\
        Par conséquent \lean{apply l_not_in_K at hl'} change \lean{hl'} en \lean{False}.\\
        Enfin quand une hypothèse est égale au but, il suffit d'appliquer la tactique \lean{exact}, ce qui termine la preuve.
\end{itemize}

\vspace{\baselineskip}
Revenons sur l'hypothèse \lean{diff}, le but est la négation d'une égalité, donc la tactique \lean{intro} introduit l'hypothèse de l'égalité et change le but en \lean{False}. Ceci nous permet de résonner par l'absurde. Ensuite, le lemme de la fermeture de l'adhérence appliqué à \lean{K} et nommé \lean{j}.\lean{rw} remplace alors \lean{Closure K} par \lean{K} dans \lean{j}. D'où une contradiction avec \lean{h}. Par conséquent \lean{apply h at j} change \lean{j} en \lean{False}. Encore une fois, \lean{j} est exactement le but, donc \lean{exact j} termine la preuve. L'hypothèse suivante \lean{c} est déduite de \lean{diff} par \lean{Set.ssubset_iff_subset_ne}, un lemme de mathlib utilisé dans le sens indirect (\lean{.mpr}). Il prend en argument \lean{sub_closure}, l'inclusion de tout ensemble dans son adhérence, en particulier pour \lean{K} (argument implicite), puis \lean{diff.symm} où \lean{.symm} permet d'inverser les membres de gauche et de droite. Enfin l'hypothèse \lean{l_in_closure_not_in_K} est déduite de l'inclusion strict de \lean{c} par le lemme de Mathlib \lean{Set.exists_of_ssubse}.\\

Les preuves suivantes seront moins détaillées afin d'éviter de trop se répéter, tous les théorèmes commençant par \lean{Set.} et \lean{Function.} proviennent de Mathlib. Poursuivons avec la proposition que toute partie fermée d'une partie compacte est compacte.\\


\begin{minted}{lean}
lemma subcompact_closed_is_compact (K H: Set X) (k_compact : is_compact K) (h_sub: H ⊆ K) (h_closed : is_closed H) : is_compact H := by
  intro x
  have x_in_k : ∀ (n : ℕ), (x n : X) ∈ K := by
    intro n
    apply Set.mem_of_subset_of_mem h_sub
    apply (x n).2

  let y : ℕ → K := λ n ↦ ⟨x n, x_in_k n⟩
  obtain ⟨f, l, _, croiss_f, lim_in_K⟩ := k_compact y

  have l_in_h : l ∈ Closure H := by apply (sequential_closure H l).mpr ⟨x ∘ f, lim_in_K⟩
  rw [closure_closed_is_closed] at l_in_h
  use f, l, l_in_h, croiss_f
  have eg : ∀ n , x n = (y n : X) := by
    intro n
    rfl
  have lim_xf : lim (x ∘ f) l := by
    intro ε hε
    obtain ⟨N, hN⟩ := lim_in_K ε hε
    use N
    intro n hn
    specialize hN n hn
    have eg : x (f n) = (y (f n) : X) := eg (f n)
    rw [Function.comp_apply]
    rw [Function.comp_apply] at hN
    rw [<- eg] at hN
    exact hN

  exact lim_xf
  exact h_closed
\end{minted}

\begin{itemize}
    \item Nous montrons que \lean{H} est compact. Nous prenons donc une suite \lean{x} à valeurs dans \lean{H} et nous montrons l'existence d'une suite extraite convergente.

    \item Pour tout \lean{n}, montrer que \lean{x n} est dans \lean{K} revient à montrer que \lean{x n} est dans \lean{H}, ceci découle de l'inclusion de \lean{H} dans \lean{K}, nous changeons le but de cette manière par \lean{Set.mem_of_subset_of_mem}. Comme \lean{x n} contient la proposition d'appartenance à \lean{H} en deuxième position, nous appliquons \lean{(x n).2}\\

    \item Par la compacité de\lean{K} nous posons une suite de \lean{K} coïncidant avec \lean{x} et nous lui associons la suite extraite convergente \lean{y} composé par \lean{f}.\\

    \itemLa limite \lean{l} de cette sous-suite est dans \lean{K} mais par \lean{sequential_closure} dans le sens indirect nous montrons que \lean{l} est dans l'adhérence de \lean{H}, en effet la composé \lean{x} par \lean{f} est une suite de \lean{H} et elle converge vers \lean{l} la limite de la sous-suite de \lean{y} puisque elle coïncide avec cette dernière.\\

    \item Rappelons que \lean{closure_closed_is_closed} nous donne l'égalité de tous fermé avec son adhérence. Donc \lean{l} est dans \lean{H}.\\

    \item Finalement \lean{f} et \lean{l} sont de bons candidats pour obtenir une sous-suite de \lean{x} convergente dans \lean{H}.\\

    \item Il reste à montrer que cette sous-suite converge bien vers \lean{l}. On utilise alors la définition de \lean{y} pour obtenir l'égalité \lean{eg}. Puis nous obtenons la
    convergence par une preuve en $\epsilon$ et en appliquant \lean{eg}. \lean{Function.comp_apply} justifie que pour toute fonction $u$, $v$ et pour tout $x$, $(u \circ v) (x) = u(v(x))$
\end{itemize}

\begin{minted}{lean}
lemma closed_incompact_iscompact (hX : Compact X) (K : Set X) (hK : is_closed K) : is_compact K := by
  intro x
  let x' : ℕ → X := λ n ↦ x n
  obtain ⟨f, l, hf,limite⟩ := hX x'
  use f
  have eg : ∀ n , x n = x' n:= by
    intro n
    rfl
  have limite2 : lim (x ∘ f) l := by
    intro ε hε
    obtain ⟨N , hN⟩ := limite ε hε
    use N
    intro n hn
    specialize hN n hn
    rw [Function.comp_apply]
    rw [Function.comp_apply] at hN
    rw [eg]
    exact hN
  have hl : l ∈ Closure K := (sequential_closure K l).mpr ⟨x ∘ f, limite2⟩
  rw [closure_closed_is_closed hK] at hl
  use l
\end{minted}

\vspace{\baselineskip}

L'image d'un compact par une fonction continue est compact. En voici la preuve formalisée en Lean.

\begin{minted} {lean}
lemma image_continuous_compact (f : X → Y) (f_continuous: Continuous f) (h_compact : Compact X) : is_compact (Set.image f Set.univ) := by

  intro y
  have hn : ∀ n, ∃ xn ∈ Set.univ, f xn = y n := by
    intro n
    exact ((Set.mem_image f Set.univ (y n)).mp (y n).2)

  choose x hx using hn
  obtain ⟨ j, l, croiss_j,lim_in_univ⟩ := h_compact x
  use j, (f l)

  have hf : f l ∈ f '' Set.univ := by
    apply (Set.mem_image f Set.univ (f l)).mpr
    use l
    apply And.intro
    exact Set.mem_univ l
    rfl

  apply And.intro
  exact hf
  apply And.intro
  exact croiss_j

  have limite : lim' (f ∘ x ∘ j) (f l) := by

    apply (sequential_continous f l).mp (f_continuous l) (x ∘ j)
    intro ε hε
    obtain ⟨ N, hN ⟩ := lim_in_univ ε hε
    use N

  intro ε hε
  obtain ⟨ N, hN ⟩ := limite ε hε
  use N
  intro n hn
  rw [Function.comp_apply]
  rw [←(hx (j n)).2]
  specialize hN n hn
  rw [ Function.comp_apply] at hN
  exact hN

\end{minted}

\begin{itemize}
    \item Dans cette preuve \lean{Set.univ} désigne l'ensemble des éléments de \lean{X} et est un type de \lean{Set X}. \lean{Set.image} désigne l'image par \lean{f} d'un type de \lean{Set X}. Donc \lean{Set.image} ne peux pas prendre \lean{X} en argument. On prendra donc l'image de \lean{Set.univ}.\\\\

    \item Comme précédemment, nous montrons qu'un ensemble est compact en introduisant une suite, il faut alors construire une sous suite qui converge dans cette ensemble. Ici l'ensemble est \lean{Set.image f Set.univ}, ainsi par définition de l'image direct, nous construisons d'abord la suite \lean{x}.

    \itemLa tactique \lean{obtain} introduit une sous suite de \lean{x} convergente, dont l'existence est déduite de \lean{h_compact x}.

    \item La tactique \lean{use} propose \lean{j} pour extraire une sous suite de \lean{y} par composition et \lean{f l} pour la limite de cette sous suite. Le but se réduit à vérifier que \lean{j} et \lean{f l} sont de bons candidats.

    \item On montre que \lean{f l} est dans \lean{(Set.image f Set.univ}. Cette propriété est vrai pour tout élément de \lean{X}. 

    \item Avant d'appliquer cette propriété, \lean{apply And.intro} permet de séparer les arguments du but \lean{ ⊢ f l ∈ f '' Set.univ ∧ strictement_croissante j ∧ lim (y ∘ j) (f l)}. On fait de même pour appliquer la stricte croissance de \lean{j}. 

    \item Enfin, la convergence de la sous-suite de \lean{y} est obtenu par continuité de \lean{f} en écrivant \lean{y} comme la composée de \lean{f} par \lean{x}. Cependant, \lean{limite} est prouvé pour \lean{lim'} car la sous-suite écrite sous cette forme est à valeur dans \lean{Y}. Or nous voulons qu'elle converge dans \lean{Set.image f Set.univ} de type \lean{Set Y}.

    \item Ainsi, on montre que \lean{lim} coïncide trivialement avec \lean{lim'} par une preuve en epsilon.

\vspace{\baselineskip}

Terminons avec un dernier résultat sur la complétude: tout espace métrique compact est complet.

\begin{minted} [breaklines] {lean}
lemma compact_is_complet (K : Type v) [MetricSpace K] : Compact K -> Complet K := by
  intro h x hx
  obtain ⟨f, l, hf,lim_l⟩ := h x
  use l
  apply Cauchy_val_adherence_conv x l f hf lim_l hx
\end{minted}

La preuve est simple. Il suffit d'introduire une suite \lean{x} de Cauchy. Comme \lean{K} est compact (\lean{h}), \lean{x} admet une sous-suite qui converge. On introduit cette sous-suite ainsi que sa limite nommée \lean{l} par la tactique \lean{obtain}. On propose \lean{l} comme candidat de la limite de \lean{x}. Enfin, ceci est vérifié par le lemme \lean{Cauchy_val_adherence_conv} démontré dans la section Complétude.

\vspace{\baselineskip}

\subsection{Compacité et recouvrements}

Pour tout compact et pour tout rayon \lean{α} il existe un recouvrement fini par des des boules de rayon \lean{α}. Cette propriété est appelé précompacité. Pour le démontrer l'idée est de raisonner par contraposée. En supposant l'existence d'un recouvrement fini par des boules de rayon \lean{alpha} , on peut construire une suite dont tout les termes sont séparés d'une distance supérieur ou égale à \lean{alpha}. Une telle suite ne peut admettre de valeurs d'adhérence. La partie la plus complexe de la formalisation de ce lemme en Lean a été de construire cette suite par récurrence. Commençons donc par un exemple plus simple d'un tel procédé, "le petit lemme" \lean{h_suite}:

\begin{minted}{lean}
lemma h_suite (X: Type) (H : (∀ A : Finset X, ∃ x : X, x ∉ A)) : (∃ u : ℕ → X, ∀ k l, k < l → (u k ≠ u l)) := by
  classical
  choose new h_new using H
  let f := λ (s : Finset X) ↦ insert (new s) s
  let f_set := λ n ↦ Nat.iterate f n ∅
  have h_rec : ∀ n : ℕ , f_set (n + 1) = f (f_set n) := by
    simp only [f_set]
    intro n
    apply Function.iterate_succ_apply' f

  use new ∘ f_set
  intro k l h_kl

  have h_in: new (f_set k) ∈ f_set l := by
    induction l with
    | zero => cases h_kl
    | succ n ih =>
      rw [h_rec]
      simp only [f]
      have h := lt_or_eq_of_le (Nat.lt_succ_iff.mp h_kl)
      rw [Finset.insert_eq]
      cases h with
      | inl h1 =>
        apply ih at h1
        exact Finset.mem_union_right {new (f_set n)} h1

      | inr h2 =>
        rw [h2]
        exact Finset.mem_union_left (f_set n) (Finset.mem_singleton_self (new (f_set n)))
  rw [Function.comp_apply]
  rw [Function.comp_apply]
  intro j
  rw [j] at h_in
  apply h_new ( f_set l) at h_in
  exact h_in
\end{minted}


\begin{itemize}

    \item \textbf{Mise en place des hypothèses}

    La tactique \lean{choose} crée à partir de \lean{H}, une fonction qui à tout ensemble fini associe un nouvel élément. Nous nommons \lean{new} cette fonction et \lean{h_new} la propriété qu'elle vérifie.

    Ensuite, nous construisons \lean{f} qui a tous ensemble fini \lean{s} associe la réunion de \lean{s} et de \lean{{new(s)}}. Cette fonction est bien défini car nous avons commencé la preuve par \lean{classical}, ceci nous a permis d'utiliser l'axiome du choix dans la construction de \lean{new}.

    Enfin nous définissons la suite \lean{f_set} par itération de \lean{f}. Elle est initialisé à \lean{0} par \lean{(f_set 0) = (f \empty)} puis pour tout \lean{n}, \lean{(f_set n+1) = f (f n) }. On obtient donc une suite d'ensembles croissante pour l'inclusion, ou chaque ensemble contient un nouveau terme.

    Nous nommons enfin \lean{h_rec} la relation de récurrence vérifiée par cette suite.

    \item \textbf{Résolution du goal }
    
    Rappelons que notre but est \lean{∃ u, ∀ (k l : ℕ), k < l → u k ≠ u l}. Nous devons alors montrer que la fonction composée \lean{new ∘ f_set} convient. Nous fixons \lean{k},\lean{kl} tels que \lean{ k<l} et par récurrence sur \lean{l} nous montrons \lean{h_in: new (f_set k) ∈ f_set l}. Ensuite, \lean{intro j} suppose l'égalité \lean{new(f_set k)=new (f_set l)} et le but est alors de montrer que c'est faux. La tactique \lean{rw} remplace \lean{new (f_set k)} par \lean{new (f_set l)} dans \lean{h_in}. Nous obtenons alors une contradiction avec \lean{ h_new ( f_set l) }.
\end{itemize}

A présent voici le lemme qui nous intéresse :

\begin{minted}{lean}
lemma recouvrement_fini (hX : Compact X) (α : ℝ) (hα : α > 0) : ∃ n, ∃ x : Fin n → X, Set.univ ⊆ (⋃ xi ∈ List.ofFn x, B(xi, α)) := by
  contrapose hX
  push_neg at hX
  have h : ∀n, ∀ xn : Fin n → X, ∃ un ∈ Set.univ, un ∉ ⋃ xn_i ∈ List.ofFn xn, B(xn_i,α) := by
    intro n xn
    apply Set.not_subset_iff_exists_mem_not_mem.mp (hX n xn)
  unfold Compact
  push_neg
  have H : ∀ A : Finset X, ∃ a ∈ Set.univ, a ∉ ⋃ xn_i ∈ A, B(xn_i,α) := by
    intro A
    let s := Finset.toList A
    specialize h s.length
    let x : Fin s.length → X := λ n ↦ s.get n
    specialize h x
    obtain ⟨a, ha, hna⟩ := h
    use a
    apply And.intro
    exact ha
    push_neg
    have eg1 : s = List.ofFn x := by
      apply List.ext_get
      rw [List.length_ofFn]
      intro n h1 h2
      simp [x]
    rw [<- eg1] at hna
    simp [s] at hna
    simp
    exact hna

  have h': ∃ u : ℕ → X, ∀ k l, k < l → (u k ≠ u l) ∧ d (u k) (u l) >= α := by
    classical
    choose new h_new using H
    let f := λ (s : Finset X) ↦ insert (new s) s
    let f_set := λ n ↦ Nat.iterate f n ∅
    use new ∘ f_set
    intro k l h_kl
    have h_rec : ∀ n : ℕ, f_set (n + 1) = f (f_set n) := by
      simp only [f_set]
      intro n
      apply Function.iterate_succ_apply' f
    have h_in : new (f_set k) ∈ f_set l := by
      induction l with
      | zero => cases h_kl
      | succ n ih =>
        rw [h_rec]
        simp only [f]
        have h := lt_or_eq_of_le (Nat.lt_succ_iff.mp h_kl)
        rw [Finset.insert_eq]
        cases h with
      | inl h1 =>
        apply ih at h1
        exact Finset.mem_union_right {new (f_set n)} h1
      | inr h2 =>
        rw [h2]
        exact Finset.mem_union_left (f_set n) (Finset.mem_singleton_self (new (f_set n)))
    simp
    push_neg
    apply And.intro
    intro j
    rw [j] at h_in
    have g : (f_set l : Set X) ⊆ ⋃ xn_i ∈ f_set l, B(xn_i, α) := by
      rw [Set.subset_def]
      intro t ht
      simp
      use t
      apply And.intro
      exact ht
      rw [dist_sep_eq_zero]
      exact hα
    apply Set.not_mem_subset g (h_new (f_set l)).2 at h_in
    exact h_in
    have not_in_ball: (new (f_set l)) ∉ B(new (f_set k), α):= by
      have h := (h_new (f_set l)).2
      simp
      simp at h
      exact h (new (f_set k)) h_in
    simp at not_in_ball
    exact not_in_ball

  obtain ⟨x, hx⟩ := h'
  use x
  intro f l hf h_lim
  have h_cauchy := Converg_Cauchy (x ∘ f) h_lim
  specialize h_cauchy α hα
  obtain ⟨N, hN⟩ := h_cauchy
  specialize hN N (Nat.le_refl N) (N+1) (Nat.le_succ N)
  simp at hN
  unfold strictement_croissante at hf
  have contradiction:= (hx (f N) (f (N+1)) (hf (N+1) N (Nat.lt_succ_self N))).2
  linarith
\end{minted}

Commentons cette longue preuve et voyons comment intervient le "petit lemme" précédent. Nous précisons que \lean{Fin n} correspond à l'ensemble des entiers naturels compris entre \lean{0} et \lean{n}, \lean{List.ofFn x} correspond à la liste contenant les valeurs ordonnées prise par \lean{x}. Le code est divisé en trois parties par des saut de lignes. Expliquons chacune d'elle brièvement :

\begin{enumerate}
    \item On résonne par contraposée. L'hypothèse \lean{h} se déduit alors simplement de \lean{Hx} qui est la négation du but. Elle nous permet d'établir \lean{H} qui prend en argument un ensemble fini au lieu d'une famille fini. C'est donc seulement une réécriture plus simple de \lean{h}. Pour ce faire, nous convertissons l'ensemble \lean{A} en liste par \lean{Finset.toList}. Nous posons une suite de \lean{Fin s.length} dans \lean{X}, où \lean{s.length} est la longueur de la liste. Ces éléments nous permettent alors d'utiliser \lean{h}, et la propriété obtenue doit être adaptée pour correspondre à notre but. Ainsi en quelques étapes nous remplaçons \lean{List.ofFn x} par \lean{A}.
    \item Comme prévu l'hypothèse \lean{H} nous permet d'obtenir \lean{h'} ! Il suffit simplement de reprendre la preuve du "petit lemme" en l'adaptant à notre contexte...
    \item Enfin nous montrons que la suite obtenue n'admet pas de valeur d'adhérence.
\end{enumerate}

Ici s'achève nos travaux sur la compacité, ce dernier lemme pourrait être utilisé pour prouver le théorème suivant: un espace métrique est compacte si et seulement si il est complet et précompact.

\newpage
\section{Conclusion}

\subsection{Perspectives d’approfondissement}

Nos travaux sur la compacité et la complétude peuvent encore être enrichis de preuves supplémentaires, nous en ajouterons probablement quelques une dans notre répertoire git. Travailler plus en profondeur sur les manuels nous permettrait également de revoir l’ensemble de notre code en le rendant plus soigné et plus concis. Enfin, pour aller plus loin, notre code pourrait être adapté au logiciel D$\exists\forall$DUCTION.


\subsection{Intérêt pédagogique}

Programmer en Lean nous a permis de prendre conscience de l’importance des hypothèses dans une démonstration. Cela nous a aussi aider à structurer notre raisonnement. En effet, l’interface de l’éditeur de code permet de visualiser chaque étape du raisonnement et nous offre un retour instantané sur la validité des arguments. Ainsi nous pouvons nous corriger facilement. Enfin, Lean permet d'explorer les fondements logiques des mathématiques, de mettre en lumière ces choses sous-entendus ou implicites des mathématiques habituelles, ainsi que les axiomes de base et les règles d'inférence.


\subsection{Compétences acquises grâce au travail de groupe}

Nous avons découvert l’importance d’une bonne communication lors d’un travail en groupe. Les réunions régulières nous permettaient de faire le point sur l’avancement de chacun. Nous communiquions aussi à distance pour demander de l’aide ou en apporter. Cependant nous avons choisi dès le début de répartir le code entre nous trois. Cette répartition aurait été plus efficace si nous avions préalablement appris ensemble a coder. La prise en main de Lean s’est en effet étalé sur une très longue période car chacun apprenait de son coté. Nous aurions alors pu aller encore plus loin dans la formalisation de la topologie en Lean.

\newpage

\appendix

\section{Méta programmation}

Une \textbf{monade} est une abstraction en programmation fonctionnelle. C'est une structure ayant pour fonction d'encapsuler des données dans une boîte elle même porteuse d'informations. Quelques exemples usuels :

\begin{itemize}
    \item \haskell{Maybe} doté de deux constructeurs \haskell{Just} et \haskell{Nothing} tel que \haskell{Just} prenne un argument, permet de représenter des procédures ne renvoyant pas forcément une valeur.
    \item \haskell{State s} qui représente une procédure avec un état de type \haskell{s}, que l'on peut modifier au cours de la procédure.
    \item \haskell{Reader} qui représente une procédure ayant accès à un environnement (ou contexte) en lecture seule.
\end{itemize}

En Lean, une tactique peut donc être vu comme une instruction qui manipule \textit{l'état} de la preuve et ayant accès à un \textit{contexte}. Une tactique est représentée comme une monade \lean{TacticM}, c'est à dire \lean{ReaderT Context $ StateRefT State TermElabM} (on dit que \lean{TacticM} étend \lean{TermElabM} avec un état et un contexte).

On détaille ici le code de la tactique \lean{custom_assump} (voir section \ref{code:custom_assump} pour l'idée générale) :

\begin{minted}{lean}
elab "custom_assump" : tactic =>
  Lean.Elab.Tactic.withMainContext do
    let goal ←Lean.Elab.Tactic.getMainGoal
    let goalType ←Lean.Elab.Tactic.getMainTarget
    let ctx ←Lean.MonadLCtx.getLCtx
    let option_matching_expr ←ctx.findDeclM? fun decl: Lean.LocalDecl => do
      let declExpr := decl.toExpr
      let declType ←Lean.Meta.inferType declExpr
      if ←Lean.Meta.isExprDefEq declType goalType
        then return Option.some declExpr
        else return Option.none
    match option_matching_expr with
    | some e => Lean.Elab.Tactic.closeMainGoal e
    | none =>
      Lean.Meta.throwTacticEx `custom_assump goal
        (m!"unable to find matching hypothesis of type ({goalType})")
\end{minted}

\begin{itemize}
    \item On commence par définir la tactique \lean{custom_assump} avec l'annotation \lean{tactic}, qui classifie la syntaxe "\lean{custom_assump}" dans la catégorie syntaxique \lean{tactic}. Le corps de la tactique est ce qui suit \lean{=>} : un terme de type \lean{TacticM Unit}.
    \item La première instruction, \lean{Lean.Elab.Tactic.withMainContext} place la procédure qui suit dans le contexte de la preuve en cours.
    \item On peut alors obtenir du contexte le but et le type du but, c'est à dire la proposition à prouver (lignes 3 et 4)
    \item On peut aussi obtenir la liste des hypothèses (ligne 5), stockées dans une structure appelée \lean{LocalContext}. On extrait a liste des déclarations (de type \lean{LocalDeclarations}) à travers laquelle on itère (ligne 6). Pour chaque déclaration :
        \begin{itemize}
            \item On obtient l'expression de la déclaration (ligne 7) et on infère son type (ligne 8)
            \item On peut alors vérifier le ce type est définitionnellement égal au type du but (ligne 9), et si c'est le cas on peut retourner l'expression (ligne 10).
        \end{itemize}
    \item On vérifie alors si l'une des expressions a bien le bon type. Si l'itération a retourné une expression, on clôt le but avec celle ci (ligne 13). Sinon, on renvoie une erreur avec le nom de la tactique, le but et le message à afficher.
\end{itemize}

\newpage

\selectlanguage{french} % Doesn't work ???
\nocite{*}
\bibliographystyle{plain}
\bibliography{main}

\end{document}
